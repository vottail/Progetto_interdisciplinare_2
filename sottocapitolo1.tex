\subsubsection{Definizione di infibulazione}
\paragraph{Di cosa si tratta}


L’infibulazione, o meglio conosciuta come mutilazione genitale femminile, è una pratica effettuata da alcuni popoli africani e asiatici che ha come scopo principale quello di non concedere alle ragazze di avere rapporti sessuali. Vi sono diversi tipi di mutilazioni e alcuni di questi hanno delle conseguenze gravi che, talvolta, si ripercuotono soltanto in futuro. (Treccani: infibulazione; Mutilazioni-genitali-femminili). Nel 1995, a Ginevra fu organizzata dall’Organizzazione Mondiale della Sanità (OMS) una giornata per condannare le mutilazioni genitali femminili e venne data una definizione: “Tutte le pratiche che comportano una rimozione dei genitali femminili esterni e/o altri danni agli organi genitali femminili per ragioni culturali o di altra natura, non terapeutica.” (Morrone, 2017; Diritto.it). Nella storia antica, il termine utilizzato era circoncisione, mentre col passare del tempo, fu coniato il termine mutilazione che richiama l’attenzione all’opinione pubblica e rende l’idea di qualcosa di intollerabile, un concetto di indignazione e di denuncia culturale.


\paragraph{Dati statistici}
I dati statistici parlano chiaro, oggi fin troppe donne sono infibulate. Il Parlamento europeo ha stimato che 500.000 tra donne e bambine che vivono in Europa soffrono le conseguenze di questa pratica e 180mila sono a rischio ogni anno, infatti tante ragazze vengono mandate nel loro paese d’origine durante le vacanze estive e sono costrette a subire tale usanza (Morrone, 2017). Per quanto riguarda la Svizzera, sono circa 15mila le donne interessate e, secondo l’UNICEF, vengono violati i loro diritti fondamentali anche su suolo elvetico. Complessivamente, le donne infibulate sono circa 200 milioni, di cui 44 milioni sono minori di quindici anni. In Gambia, il 56 per cento delle bambine non aveva ancora compiuto undici anni. Negli Stati Uniti, secondo il Centers for Disease Control and Prevention di Atalanta (CDC), il numero di casi è triplicato e il motivo principale sono le immigrazioni. Tuttavia, nel corso degli ultimi 15-20 anni, in alcuni paesi c’è stata una notevole diminuzione di questo fenomeno. Come si può vedere nella figura X, il Sudan, la Somalia, il Mali e l’Egitto, sono i paesi con un numero maggiore di mutilazione genitale femminile. La maggior parte, si concentra in Africa, mentre in Asia, è l’Indonesia la parte più colpita. 
IMMAGINE

