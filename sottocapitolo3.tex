
\subsubsection{Cause e conseguenze di questa procedura}

Vi sono varie ragioni che spingono le popolazioni a praticare le mutilazioni dei genitali femminili, e sono: quella socioculturale, in cui vi è la credenza che la donna diventa matura solamente dal momento in cui le si asportano la clitoride e che mantenendo intatti gli organi genitali non potrebbe rimanere vergine. Le donne che non subiscono la mutilazione dei genitali vengono considerate da tutti come impure e ciò scredita le famiglie e non permette loro di trovare marito in futuro; infatti questa pratica prepara al matrimonio e fa passare all’età adulta, fa quindi parte dell’educazione. Ciò che è bene precisare è che, molte delle ragazze mutilate, non vengono obbligate, bensì, fin da piccole sono influenzate dalle donne anziane, che hanno un ruolo importante e molto rispettato nella comunità e gli viene dunque insegnato che le sofferenze subìte sono parte dell’essere donna, vengono dunque istruite con questa ideologia. 
Oltre alla ragione socioculturale, vi è quella estetica ed igienica, in cui viene associata ai genitali esterni della donna un’idea di bruttezza e si crede che con il passare del tempo possano crescere e condizionare la vita e la salute. Oltre a ciò, pensano che producano un cattivo odore. Alcuni leader religiosi sono a favore alle MGF, nonostante ciò, nessuna religione richiede tale pratica e viene eseguita anche da comunità che hanno un credo differente. Vi sono però allo stesso tempo dei capi religiosi che lottano per la sua eliminazione e sostengono che non siano assolutamente le religioni ad imporla. A sostegno delle mutilazioni vi sono motivi pure sessuali e psicologici che si giustificano con fatto che le donne infibulate abbiano molto meno l’istinto e la voglia di avere delle relazioni extraconiugali, o ancora prima, di perdere la verginità prima del matrimonio e inoltre, si pensa che le donne infibulate abbiano meno voglia di compiere degli atti sessuali perché sanno che gli causerebbe molto dolore (Diritti umani). 
Addirittura, si pensa che la clitoride possa uccidere il bambino durante il parto solamente con il contatto. Si crede dunque che mutilare le donne sia una sorta di prevenzione alle morte prenatali. Tuttavia, è esattamente il contrario: chi ha subìto una circoncisione rischia maggiormente di avere problemi durante il parto (Diritti umani).

\paragraph{Aspetti sociosanitari}
Per conoscere le mutilazioni genitali femminili a fondo, occorre sapere che vi sono diverse classificazioni, che si sono modificate col passare del tempo. In seguito, verranno elencate e spiegate. 
\begin{itemize}
    \item Tipo I: Questo tipo di circoncisione femminile viene chiamato clitoridectomia o anche Sunna nei paesi islamici, ovvero “tradizione”, consiste nell’escissione circonferenziale del prepuzio e prevede una piccola incisione sul prepuzio del clitoride senza però asportarne nessuna parte. 
	\item{Tipo II: In questa seconda tipologia, denominata anche escissione, la rigidità varia notevolmente, essa consiste solitamente nell’amputazione della clitoride (o solamente del glande o intera) e l’asportazione delle piccole labbra.}
I primi due tipi rappresentano genericamente l’80-85\% dei casi di infibulazione.
\item {Tipo III: Il terzo tipo di circoncisione, in Sudan viene chiamato anche circoncisione faraonica e circoncisione sudanese in Egitto. Consiste nella chiusura parziale dell'apertura vaginale dopo aver escisso una parte di tessuto vulvare. La pratica più estrema comprende la rimozione della clitoride intera e delle piccole labbra. I due lati delle grandi labbra vengono poi cuciti tra di loro utilizzando spine di acacie, impacchi e punti di sutura e le gambe vengono poi legate insieme per 2-6 settimane. L’unica apertura che rimane misura dai 2 ai 3 centimetri, o addirittura può arrivare ad avere la grandezza dell’estremità di un fiammifero. Questo per far sì che la donna possa urinare e per permettere l’uscita del sangue mestruale. Se, dopo la cicatrizzazione, l’orifizio rimasto non è abbastanza piccolo, l’operazione viene ripetuta. Questa terza classificazione è l’unica che si può definire infibulazione, mentre le altre due sono chiamate solamente mutilazioni dei genitali. In Sudan, Somalia e Djibouti vi è una percentuale dell’80-90\% che ha subìto tale pratica, mentre per gli altri paesi in cui viene effettuata, la percentuale è ridotta ed è del 15-20\% in media.}
\item {Tipo IV: Il quarto ed ultimo è quello più crudele, esso comprende procedure molto diverse tra loro, due di esse sono le più importante e consistono nel raschiamento del tessuto circostante l’apertura vaginale per quanto riguarda il primo e nell’incisione della vagina nella parte posteriore. Queste due procedure, denominate angurya cuts e gishiri cutting o cuts vengono praticate per la maggior parte dei casi in Nigeria dai popoli delle regioni di Hausa e Fulani.}
Nella figura X si può vedere un’illustrazione dei diversi tipi di mutilazioni. 
\end{itemize}
IMMAGINE!!!!

Ciò che si sa con certezza è che le mutilazioni genitali femminili non prestano alcun vantaggio per le bambine e le donne adulte. Le conseguenze della pratica sono molteplici e dipendono dal tipo di mutilazione, solitamente l’infibulazione è la più rischiosa (tipo III e IV). Si suddividono in due: fisiche e psichiche/sessuali/sociali. Gli effetti fisici immediati sono: lo shock emorragico o neutrogeno, che possono portare al decesso. Mentre a breve termine possono esserci delle infezioni causate dalla scarsità di igiene durante l’operazione, oppure anche un’emorragia per via dell’esportazione della clitoride e a sua volta del coinvolgimento dell’arteria clitoridea. 
Vi sono anche delle conseguenze tardive, che vanno dal cheloide, ovvero un’infezione a causa di una guarigione incompleta della ferita, alla difficoltà nella minzione per il danno all’apertura uretrale, varie cisti che possono formare un tumore e difficoltà mestruali come conseguenza dell'ostruzione dell’orifizio vaginale. Altre conseguenze sono per esempio la sterilità, causata dal danneggiamento degli organi riproduttivi, l’accumulo di detriti mestruali che porta poi alla formazione di calcoli nella vagina. Si possono verificare anche problemi gravi durante cure ginecologhe, come pure durante la gravidanza o il parto, durante il quale bisogna incidere chirurgicamente l’orifizio per facilitare il parto. Tutte queste conseguenze sono fisiche e danneggiano la maggior parte delle donne mutilate, a dipendenza della gravità e del tipo di pratica effettuata. 
Per quanto riguarda gli aspetti psichici, sessuali e sociali, la donna potrebbe avere difficoltà nel raggiungere l’orgasmo, avere una scarsa sensibilità agli stimoli sessuali a causa dell’asportazione del clitoride e l’incapacità di avere rapporti sessuali. Tutti questi fattori messi insieme potrebbero causare depressione, ansia, incubi, malattie psicosomatiche ed effetti negativi che durano per tutta la vita. La maggior parte di coloro che subiscono tale operazione cresce con la paura, con un’esperienza di sottomissione, inibizione e soppressione di sentimenti. (Diritti Umani; Uefgm; Sante Sexuelle).
