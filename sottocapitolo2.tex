\subsubsection{Cenni storici della mutilazione genitale femminile}
\paragraph{Quali sono le culture e le nazioni che impongono questa pratica?} 

La mutilazione genitale femminile ha un’origine sconosciuta, poiché non vi sono testimonianze certe che indicano come e quando la pratica sia nata e in che modo si sia diffusa. Anche se essa è una pratica tradizionale, mettendo a confronto testi documentali e linguaggi, come anche classificazioni, è difficile e non sempre è possibile ricostruire una linea del tempo coerente. Tuttavia, storici e filosofi, hanno stilato delle supposizioni sulla sua origine. Una parte degli studiosi ha constatato che deriva da un luogo nella penisola araba o dall’Egitto, infatti in alcuni paesi prende il nome di “circoncisione faraonica”. Mentre altri ricercatori credono che sia originaria di posti diversi e si sia sviluppata in momenti storici differenti tra loro. 
Un’antica leggenda, racconta di una donna che regnava diverse zone dell’attuale Somalia, chiamata Araweelo. Essa, essendo molto potente, castrava tutti i maschi del suo regno, pensando che in quel modo sarebbe stato più facile sottoporli alle sue leggi. Dopo qualche anno, la regina fu uccisa da un suo parente e la leggenda narra che da quel momento, tutta la popolazione maschile iniziò a mutilare le donne per vendetta. 
Un altro racconto antico è quello situato fra le rive del Nilo e parla di un archeologo che nel 1862, acquistò un papiro che descriveva in dettaglio una circoncisione maschile. Si pensa che, parallelamente a questo si svilupparono le mutilazioni genitali femminili e lo si pensa anche per il modo in cui venivano fasciate e strette in prossimità dei genitali le mummie delle donne. Per quanto riguarda i greci invece, ciò che si crede è che soltanto le classi dominanti ricorressero all’aiuto di chirurghi mentre gli egizi si rivolgessero ad operatori tradizionali formati oralmente e dall’esperienza. Vi era una visione diversa della donna nelle diverse culture, per esempio in Egitto era famosa per la propensione verso la sessualità mentre negli altri paesi era vista come se sentisse la mancanza di rapporti sessuali, quindi, l’infibulazione veniva vista come una soluzione ad un problema locale egiziano (Morrone, 2017).
Anche i Romani praticavano, nei confronti degli schiavi, la mutilazione genitale femminile mediante l’applicazione di una fibula, ovvero una specie di spilla utilizzata solitamente per agganciare la toga ed il motivo era di evitare gravidanze inaspettate. 
Filone l’Ebreo e Ambrogio, due antichi autori, scrissero che sulle donne veniva praticata quando avevano all’incirca 14 anni, ovvero quando iniziava a comparire il primo ciclo mestruale, mentre l’uomo veniva circonciso prima del matrimonio. 
Le procedure chirurgiche vennero evidenziate per la prima volta da Sorano d’Efeso, medico greco, che scrisse un trattato di ginecologia. Il suo testo fu il riferimento principale su questo tema, fino alla pubblicazione del Giardino delle rose di Eucario Rodione nel 1513. Oltre a credere che si praticasse in Egitto, Sorano scrisse che era presente anche in Lidia. Successivamente, nella metà del II secolo dopo Cristo, Galeno ed Ezio, scrivono un resoconto completo della procedura, che arriva fino ai giorni nostri (Morrone, 2017; Diritto.it). 
