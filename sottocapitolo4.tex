\subsubsection{Aspetti legali da prendere in considerazione}

Nel mondo, vi sono molte diversità e al giorno d’oggi non viviamo completamente in una realtà multietnica, bensì, la cultura è eurocentrica, antropocentrica e androcentrica. Con l’immigrazione, oltre alle diversità di genere, vi sono quelle culturali, infatti i valori occidentali vengono esaltati maggiormente. Samuel Huntigton, esperto di politica estera, afferma che la tradizione di diritti e la libertà tipici dell’Occidente sono considerati superiori. Il fenomeno migratorio ha reso più visibile questa differenza e le mutilazioni dei genitali femminili hanno un ruolo, non di impatto massmediologico ma di attuale importanza.
Le mutilazioni genitali femminili sono, per molte comunità, una pratica difficile da abbondare perché fanno parte della cultura e della tradizione. Per eliminare ciò, è necessario che un grande numero di famiglie scelga di interrompere la pratica, affinché nessuna bambina o ragazza si senti esclusa e sia svantaggiata. Abbandonando la pratica in modo esplicito, le famiglie sono convinte che anche le altre la cessino in modo definitivo e dunque non ci si sente gli unici della comunità. In questo modo, la norma sociale presente in tutte le popolazioni che mettono in atto questa tradizione, permetterà ugualmente alle bambine di sposarsi senza essere giudicate. 
Nel 1931, durante una conferenza tenutasi a Ginevra dalla Società per la salvaguardia dell’infanzia, si chiese per la prima volta l’abolizione delle mutilazioni genitali femminili. Invece di intervenire sulla pratica stessa, la proposta fu quella di puntare sullo sviluppo dell’istruzione, in modo da informare la popolazione e permettere di riflettere sulla conservazione o il rigetto dell’usanza. 
Solamente alla fine degli anni cinquanta, l’Organizzazione mondiale della sanità (OMS), si rese conto del problema e cominciò ad attivarsi. Nonostante la volontà dell’OMS, dopo aver riconosciuto tali operazioni rituali come Mutilazioni Genitali Femminili (MGF), vennero definite come pratiche legate a concezioni sociali e culturali, quindi di non di competenza medica. Dopodiché, nel 1975, ovvero l’Anno internazionale della donna, fu aperto un decennio di discussione sulla condizione femminile. E molte femministe fecero per la prima volta una statistica stimata sulla situazione delle operazioni dei genitali delle donne e nel 1979, venne definito come un problema di salute pubblica. In questo modo, dopo diversi appuntamenti internazionali in differenti paesi, si creò il Comitato interafricano sulle pratiche tradizionali pregiudizievoli per la salute delle donne e dei bambini (Iac) con sede a Dakar, in Senegal. Grazie ad essa, ci fu un vero e proprio cambiamento di paradigma e si iniziarono a formulare le prime campagne globali. All’inizio degli anni novanta, vennero concepite come una violazione dei diritti umani internazionali, come pure, negli anni a seguire, come violazione dei diritti delle bambine e delle donne e successivamente, nel 1993 a Vienna, furono classificate all’interno della “violenza contro le donne”. Nel 1995, le si pose sullo stesso piano della violenza sessuale. 
La Convenzione di Istanbul, in Europa, prevede la protezione delle donne e l’eliminazione di qualsiasi tipo di violenza quali fisica, sessuale, psicologica, matrimonio forzato, aborto forzato, sterilizzazione forzata, omicidi, come pure delle mutilazioni genitali femminili. 
La Commissione Europea, nel 2013, propose quattro aree di intervento:
\begin{itemize}
\item{Prevenzione mutilazioni genitali femminili attraverso cambiamenti sociali duraturi}
\item{Supporto degli Stati membri per un’azione pena}
\item{Promozione dell’eliminazione della pratica nel mondo}
\item{Valutazione, monitoraggio e attuazione di metodi per l’eliminazione}
\end{itemize}
Per quanto riguarda gli aspetti legali in Svizzera, nel 2005, Maria Roth-Bernasconi, consigliera nazionale, ha lanciato un’iniziativa parlamentare chiedendo l’introduzione di una norma che punisce la pratica diretta e l’incitazione a operare mutilazioni di organi genitali femminili in Svizzera. Successivamente, il 12 febbraio 2009, la Commissione degli affari giuridici valutò il caso e constatò che il Codice penale svizzero (CP), non conteneva disposizioni che potessero espressamente punire chiunque commettesse tale operazione, bensì che chi commettesse il reato, veniva sanzionato attraverso le norme che tutelano l’integrità della persona. Ci fu quindi una discussione per modificare l’articolo 122 CP, in cui veniva esplicitato che chiunque intenzionalmente ferisce una persona mettendone in pericolo la vita o gli infligga ferite dello stesso grado di gravità, è perseguito d’ufficio con una pena detentiva sino a dieci anni o con una pena pecuniaria non inferiore a 180 aliquote giornaliere (Admin). La Commissione, dopo avere discusso, ritenne che le sanzioni erano insufficienti e chiese dunque l’introduzione di una norma specifica, a dipendenza del luogo dell’operazione. Venne dunque proposta la modifica dell’articolo 122 del Codice penale svizzero e venne dichiarato anche che era perfettamente compatibile con le convenzioni internazionali alla quale la Svizzera aderiva. 
Dopodiché, nel 2012, fu integrato un articolo che vietasse in modo esplicito tale pratica. Ed esso cita: 
“1 Chiunque mutila gli organi genitali di una persona di sesso femminile, pregiudica considerevolmente e in modo permanente la loro funzione naturale o li danneggia in altro modo, è punito con una pena detentiva sino a dieci anni o con una pena pecuniaria non inferiore a 180 aliquote giornaliere. 
2 È punibile anche chi commette il reato all'estero, si trova in Svizzera e non è estradato. L'articolo 7 capoversi 4 e 5 è applicabile.” (Admin: Articolo 124, Lesioni personali / Mutilazione di organi genitali femminili). 
Ciò che fu modificato rispetto all’articolo 122, fu la parte iniziale, ovvero quella che cita “chiunque intenzionalmente ferisce”, in questo modo, è esplicito che ci si riferisce a mutilazioni femminili. Coloro che vengono puniti, sono innanzitutto l’infibulatrice, l’infibulatore o il medico, come pure i complici, ovvero i genitori o parenti. La complicità è presente anche dal momento in cui una persona contribuisce allo spostamento della donna da un paese all’altro per praticare l’operazione. Come scritto nel capoverso 2 dell’articolo 124, la pratica è punibile anche se effettuata nel paese natale o in uno stato qualsiasi. Quindi dal momento in cui la vittima è residente in Svizzera, anche se nel luogo dove si reca la pratica è accettata, è sanzionabile colui che opera (Uefgm).
